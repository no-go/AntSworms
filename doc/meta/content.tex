\section{Introduction (not ready)}\label{introduction-not-ready}

Raketensteuerungen von Atombomben werden oft von pneumatischen
(Staubsaugermotoren und Klappen wie bei Zuse I) oder hydraulischen
Systemen betrieben, damit sie bei einem elektromagnitischen Impuls einer
anderen Bombe noch weiter funktionieren. In der Atmospähre eines anderen
Planeten könnte eine Elektronik z.B. wegen der starken statischen Ladung
des Marsstaubs ähnliche Probleme verursachen. Unter Wasser oder in einem
sandigem Medium können GPS und Funkwellen kaum bis gar nicht genutzt
werden. Schaut man sich aktuelle Nanotechnologien an, so werden
chemische Botenstoffe zur Auslösung bestimmter Reaktionen,
interessanter, als mit Funktechnik Roboter in unserer Blutbahn
kommunizieren zu lassen. In der Natur werden bereits Hormone (und keine
Funkwellen) als Kommunikation eingesetzt. Sollte man chemische Propeller
an Makromolekülen bilden oder betreiben, die durch eine chemische
Reaktion aufgrund einer bestimmten Konzentration eines künstlichen
Hormons an der Stelle des Moleküls, an dem der Propeller sitzt, so sind
logische Operationen oder gar Speichermedien in einem Molekül begrenzt.
Durch künstliche Methabolismen eine turingvollständige Sprache bilden,
die als Eingabe Konzentrationen von bestimmten Molekülen akzeptiert und
gleiches als Ausgabemedium zur Steuerung einer molekularen Struktur
erzeugt, mag heute noch nach Science Fiction klingen - findet aber
bereits in Laboren statt. (todo: hier fehlt mir noch eine Referenz)

In contrast to bees, the simulated ants did not split into a searching
and a collecting behavior in that case.

We added a simple traffic jam control to each unit and measured a better
food supply.

This demo file is intended~\cite{Tentschert2000} to serve as demo. I wish
you the best~\cite{li2014chaos} of success. test
test~\cite{gonzalez2017smells} \ldots{}

Molecular machines, optical digital processing units and analogue
computers with a pneumatic logical unit are able to work in special
environments, where modern micro controllers are too large, use too much
electric energy or are not resistant to massive electromagnetic forces.
Many of these techniques are unable to use wireless communication,
unable to store big data sizes or do millions of instructions of a
complex algorithm. If you want to solve a big problem with it, you have
to solve it in a collective opportunistic way with a simple logic for
each unit.

Primitive Algorithmen wie den Ant-Algorithmus sind bereits seit vielen
Jahren bekannt. Dieser nutzt einen Duftstoff, der in einer Region von
jedem Teilnehmer freigesetzt wird, um eine Art kollektiven Speicher zu
verwirklichen. Ziel ist es auf diese Art Spuren für andere Teilnehmer zu
hinterlassen, welche zu Futterquellen führen. Dieses Futter soll dann
ins Nest getragen werden. Je mehr Ameisen auf einer gelegten Spur
laufen, desto stärker wird diese Spur. Theoretische Modelle zeigen, dass
sich dadurch sogar kürzere Wege zurück ins Nest durchsetzen, da diese
Wege von den Ameisen, die ihn gefunden haben, häufiger frequentiert wird
- die Duftspur wird so verstärkt. In der Literatur finden sich viele
Variationen und theoretische Modelle, wie ein solcher Algorithmus, durch
die Natur motiviert, nachgebildet werden kann. Z.B. sind lokale
Algorithmen (verteilte Algorithmen mit linearer Laufzeit, die nicht von
der Größe des Netzes abhängig sind), einem Schwarm-Algorithmus wie dem
Ant-Algorithmus sehr ähnlich, betrachten aber keine kontinuierliche
Veränderung der Daten Aufgrund von Bewegungen. Schaut man sich in der
Literatur die Ant-Algorithmen im Detail an, so wird der Erfolg einer
Ameisen-Kolonie bei gewählten Parametern, die das Legen der Duftspuren
und deren ``Verwitterung'' beeinflussen, kaum behandelt.

\section{Related Work (not ready)}\label{related-work-not-ready}

intro about solutions and our special new solution/idea. iter through
related work with focus on different solutions. Why are some aspects
open?

\section{Methods}\label{methods}

The simulation of our ant algorithm was splitted into the following
parts:

\begin{itemize}
\tightlist
\item
  a simulator (doing 1000 loops)
\item
  a world (2 dimensional)
\item
  food tiles (marked as food with a nutritive value)
\item
  pheromone tiles
\item
  30 robot objects (each with our ant algorithm)
\end{itemize}

In this chapter we explain each part and who we initialized it with
fixed values. The last chapter is about our ant algorithm and the
traffic jam control to prevent ant mills.

\subsection{Simulated parts and their
interaction}\label{simulated-parts-and-their-interaction}

We did not want to print any code here to make it easy to implement the
parts, concept and objects in your preferred system, language or medium.

\subsubsection{World}\label{world}

Our world is a 2 dimensional hex grid. Thus our robots and tiles can
only placed into that grid. The worlds are generated with 10 different
random seeds to place food tiles with different area sizes, positions
and nutritive values. When we measured a new result with a different
parameter, we simulated it with the same 10 different worlds and build a
cumulative value.

This hex grid is placed on a normal \((x,y)\) coordinate system, but the
placement of each robot and tile must fit into the hex grid. If we talk
about distances we use the euclidean distance based on that x and y
values.

\subsubsection{Robots (ant)}\label{robots-ant}

We initialized the simulator with 30 ants placed on \((0,0)\) in the
world (we call it \emph{home}). Thus they start with a traffic jam. Each
ant has a \emph{hunger level}. If an ant does not reach any food or
carries it to home, the hunger level is increment by one. If the limit
reaches \textbf{50}, it dies immediately and respawns at home. Every ant
is awaked by each simulator iteration and runs our ant algorithm with
traffic jam control.

Because of the hex grid, the ant can only move to 6 neighbor fields. For
Example if the ant starts at \((0,0)\) and is awaked by the simulator,
it can make a step to \((-1.0, 0.0)\), \((1.0, 0.0)\), \((-0.5, 1.0)\),
\((-0.5, -1.0)\), \((0.5, 1.0)\) or \((0.5, -1.0)\).

\subsubsection{Food tiles}\label{food-tiles}

Every food tile is part of an food area. In a world generated by a seed
these areas can have a size between 1 and 16 tiles. A world can have up
to \(30\) areas. Each tile is generated with a nutritive value between
\(1\) and \(20\). If an ant reaches the food tile, the nutritive value
is decremented by one. If the nutritive value is zero, the food tile is
deleted immediately. The initial placement of the areas is handled by a
random seed. There are two different concepts placing them with a
maximal area distance of \(20\) (\(\pm\) 5):

\begin{itemize}
\tightlist
\item
  spray
\item
  ordered
\end{itemize}

The spray concept setting an area on a random position and place the
next area randomly (but not to far away). Finally the sprayed areas are
centered in the middle of the world.

The ordered concept is a grid arranged around the ant home location. The
position is very by \(5\).

We decided to choose these concept to get random worlds with reachable
food for the ants. Vary the maximal distance of the ant and/or vary the
maximal distance of each area may focused in future work.

\subsubsection{Pheromone tiles}\label{pheromone-tiles}

The pheromone tiles are generated by ants, if an ant finds food and
running back to home. This is the only way creating tiles. We store the
pheromone potency internally in an alpha value with \(1.0\) as maximum.
The ants are able to increment this value. A tile with a \emph{potency}
less than \(0.0001\) will be removed by the simulator. You can define it
as a detection level of an ant, if you want.

After all ants do their work after a simulator iteration, the tiles will
be deleted or the potency will be reduced by a \emph{evaporation}:

\[ potency_{i+1} = \frac{100 - evaporation}{100} * potency_{i} \]

\subsubsection{Simulator}\label{simulator}

The Simulator loads the world with it special seed and iterated \(1000\)
times through \(30\) ants. Thus a single ant is being \(wakeup()\) for
\(1000\) times to walk, detect, collect or setting/changing a pheromone
tile.

\subsection{Ant algorithm with traffic jam
control}\label{ant-algorithm-with-traffic-jam-control}

todo \ldots{}

\section{Results (not ready)}\label{results-not-ready}

intro about selected data, getting them and how we analyse them

\subsection{Definitions and Taxonomy}\label{definitions-and-taxonomy}

Parameters

\begin{itemize}
\tightlist
\item
  different food area size
\item
  different distance between food area and home
\item
  different count of food areas
\item
  different direction random size
\item
  different prefer pattern for dusty tiles
\item
  different amount, how strong the way decisions effects on setting the
  new dust
\item
  better to add or multiply dust on a tile
\item
  loosing dust on tiles rate
\item
  different hunger limits (?)
\item
  different random seeds (?)
\end{itemize}

Effects

\begin{itemize}
\tightlist
\item
  development of collected food (\%) in 1000 iterations
\item
  used dust (sum) over 1000 iterations
\item
  how many died (how long did they life)
\item
  development of random-, jam- and dust-decisions (\%) in a simulation
\item
  distance after last 10 steps before dying (ant traps?)
\end{itemize}

\subsection{Aspect 1}\label{aspect-1}

Subsection text here.

\subsection{Aspect 2}\label{aspect-2}

Subsection text here.

\subsection{Discussion}\label{discussion}

intro, offer explanation and reference to literature

\section{Conclusion (not ready)}\label{conclusion-not-ready}

Ebenso gibt es beim Ant-Algorithmus einen Aspekt, der bei der ältesten
Art, den Wanderameisen, besondere Aufmerksamkeit verdient:
``Ameisenmülen''. In diesem Fall laufen alle Ameisen im Kreis und
sterben an Erschöpfung, da sie so keine weiteren Futerquellen finden.
Ein weiterer Punkt, der einen Erfolg eines zu primitiven Ant-Algorithmus
in Frage stellt, sind Ameisen-Straßen, die zu versiegten Futterquellen
führen. Wir konnten in unseren Simulationen beide ``Fehlentwicklungen''
beobachten.

Man kann den Ant-Algorithmus durch Hinzunahme von GPS, Funktechnik und
einem großen Arbeitsspeicher zur modifizieren, um ein solches
Fehlverhalten zu erkennen oder zu verhindern, aber diese Modifikationen
setzen folgendes voraus:

\begin{itemize}
\tightlist
\item
  elektrische Energie
\item
  Rechenleistung
\item
  mindestens die Größe des Chips
\item
  bedenkenlose Anwendung von Funkwellen
\item
  Speichermedium
\end{itemize}

Es ist also nicht vollkommen abwegig, einen Ant- Algorithmus so simple
wie möglich zu halten, der die Umge- bung als Massenspeicher nutzt,
welche mit dem Operations- gebiet mitwächst. Wir haben diesen Aspekt der
Einfachheit in Hinblick auf Erfolg und Robustheit bzgl. äußerer
Parameter untersucht und kommen zu dem Ergebnis, dass auch andere
natürliche Aspekte der Wanderameise eine wichtige Rolle Spielen müssen,
was diese Spezies so lange parallel zu anderen Ameisenarten hat
erfolgreich koexistieren lassen.

\textbf{Future Work}: new open questions? how can we find answers in the
future? How can we use our solutions in the future?
