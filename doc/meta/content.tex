\section{Introduction}\label{introduction}

blah

\section{Notes}\label{notes}

This demo file is intended~\cite{Tentschert2000} to serve as demo. I wish
you the best~\cite{li2014chaos} of success. test
test~\cite{gonzalez2017smells} \ldots{}

Building robots in hardware to test a swarm algorithm is possible, but
expensive
\href{https://en.wikipedia.org/wiki/Swarm_robotic_platforms}{hardware}

A hardware specific simulation of a single device exists, too:
\href{https://infocenter.nordicsemi.com/index.jsp?topic=\%2Fcom.nordic.infocenter.sdk52.v0.9.0\%2Fant_examples_ant_fs.html}{BLE
Ant client sim}

Using VM with router/device firmware and a network layout should be
interessting: \href{https://www.gns3.com/community}{Network sim}

Also:

\begin{enumerate}
\def\labelenumi{\arabic{enumi}.}
\tightlist
\item
  Ein kleiner teil der Bienen (Scouts) sucht global nach ``gärten''
\item
  Da Gärten, wo es viel ``Futter'' gibt ein großes Areal sind, kommen
  viele Scouts zum dance-floor zurück mit ähnlichen Koordinaten
\item
  normale Bienen folgen nun den Scouts und verbreiten sich im Areal
  (Nahsuche)
\item
  ein Teil fliegt als Scout weiter und sucht neue Gärten
\item
  durch die Nahsuche werden schlechte Regionen von den Bienen
  ausgeschlossen und bessere Regionen ``gespeichert''
\item
  Biene, die bessere Region fand, wird neuer Scout und fiegt zum dance
  floor
\item
  Biene, die in der Region keine Verbesserung nach einer Weile fand,
  sucht neue Gärten
\end{enumerate}

\hfill J. Peters \hfill \today

\section{Related Work}\label{related-work}

intro about solutions and our special new solution/idea. iter through
related work with focus on different solutions. Why are some aspects
open?

\section{Methods}\label{methods}

Describe technique, structure and data collection of our solution.

\section{Results}\label{results}

intro about selected data, getting them and how we analyse them

\subsection{Definitions and Taxonomy}\label{definitions-and-taxonomy}

more details about focused parameter and an intro to different tests and
aspects we focused in our work.

\subsection{Aspect 1}\label{aspect-1}

Subsection text here.

\subsection{Aspect 2}\label{aspect-2}

Subsection text here.

\subsection{Discussion}\label{discussion}

intro, offer explanation and reference to literature

\section{Conclusion}\label{conclusion}

The conclusion goes here.

\textbf{Future Work}: new open questions? how can we find answers in the
future? How can we use our solutions in the future?
