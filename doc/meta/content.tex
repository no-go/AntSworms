\section{Introduction}\label{introduction}

Primitive Algorithmen wie den Ant-Algorithmus sind bereits seit vielen
Jahren bekannt. Dieser nutzt einen Duftstoff, der in einer Region von
jedem Teilnehmer freigesetzt wird, um eine Art kollektiven Speicher zu
verwirklichen. Ziel ist es auf diese Art Spuren für andere Teilnehmer zu
hinterlassen, welche zu Futterquellen führen. Dieses Futter soll dann
ins Nest getragen werden. Je mehr Ameisen auf einer gelegten Spur
laufen, desto stärker wird diese Spur. Theoretische Modelle zeigen, dass
sich dadurch sogar kürzere Wege zurück ins Nest durchsetzen, da diese
Wege von den Ameisen, die ihn gefunden haben, häufiger frequentiert wird
- die Duftspur wird so verstärkt. In der Literatur finden sich viele
Variationen und theoretische Modelle, wie ein solcher Algorithmus, durch
die Natur motiviert, nachgebildet werden kann. Z.B. sind lokale
Algorithmen (verteilte Algorithmen mit linearer Laufzeit, die nicht von
der Größe des Netzes abhängig sind), einem Schwarm-Algorithmus wie dem
Ant-Algorithmus sehr ähnlich, betrachten aber keine kontinuierliche
Veränderung der Daten Aufgrund von Bewegungen. Schaut man sich in der
Literatur die Ant-Algorithmen im Detail an, so wird der Erfolg einer
Ameisen-Kolonie bei gewählten Parametern, die das Legen der Duftspuren
und deren ``Verwitterung'' beeinflussen, kaum behandelt. Ebenso gibt es
beim Ant-Algorithmus einen Aspekt, der bei der ältesten Art, den
Wanderameisen, besondere Aufmerksamkeit verdient: ``Ameisenmülen''. In
diesem Fall laufen alle Ameisen im Kreis und sterben an Erschöpfung, da
sie so keine weiteren Futerquellen finden. Ein weiterer Punkt, der einen
Erfolg eines zu primitiven Ant-Algorithmus in Frage stellt, sind
Ameisen-Straßen, die zu versiegten Futterquellen führen. Wir konnten in
unseren Simulationen beide ``Fehlentwicklungen'' beobachten.

Man kann den Ant-Algorithmus durch Hinzunahme von GPS, Funktechnik und
einem großen Arbeitsspeicher zur modifizieren, um ein solches
Fehlverhalten zu erkennen oder zu verhindern, aber diese Modifikationen
setzen folgendes voraus:

\begin{itemize}
\tightlist
\item
  elektrische Energie
\item
  Rechenleistung
\item
  mindestens die Größe des Chips
\item
  bedenkenlose Anwendung von Funkwellen
\item
  Speichermedium
\end{itemize}

Raketensteuerungen von Atombomben werden oft von pneumatischen
(Staubsaugermotoren und Klappen wie bei Zuse II) oder hydraulischen
Systemen betrieben, damit sie bei einem elektromagnitischen Impuls einer
anderen Bombe noch weiter funktionieren. In der Atmospähre eines anderen
Planeten könnte eine Elektronik z.B. wegen der starken statischen Ladung
des Marsstaubs ähnliche Probleme verursachen. Unter Wasser oder in einem
sandigem Medium können GPS und Funkwellen kaum bis gar nicht genutzt
werden. Schaut man sich aktuelle Nanotechnologien an, so werden
chemische Botenstoffe zur Auslösung bestimmter Reaktionen,
interessanter, als mit Funktechnik Roboter in unserer Blutbahn
kommunizieren zu lassen. In der Natur werden bereits Hormone (und keine
Funkwellen) als Kommunikation eingesetzt. Sollte man chemische Propeller
an Makromolekülen bilden oder betreiben, die durch eine chemische
Reaktion aufgrund einer bestimmten Konzentration eines künstlichen
Hormons an der Stelle des Moleküls, an dem der Propeller sitzt, so sind
logische Operationen oder gar Speichermedien in einem Molekül begrenzt.
Durch künstliche Methabolismen eine turingvollständige Sprache bilden,
die als Eingabe Konzentrationen von bestimmten Molekülen akzeptiert und
gleiches als Ausgabemedium zur Steuerung einer molekularen Struktur
erzeugt, mag heute noch nach Science Fiction klingen - findet aber
bereits in Laboren statt. (todo: hier fehlt mir noch eine Referenz)

Es ist also nicht vollkommen abwegig, einen Ant-Algorithmus so simple
wie möglich zu halten, der die Umgebung als Massenspeicher nutzt, welche
mit dem Operationsgebiet mitwächst. Wir haben diesen Aspekt der
Einfachheit in Hinblick auf Erfolg und Robustheit bzgl. äußerer
Parameter untersucht und kommen zu dem Ergebnis, dass auch andere
natürliche Aspekte der Wanderameise eine wichtige Rolle Spielen müssen,
was diese Spezies so lange parallel zu anderen Ameisenarten hat
erfolgreich koexistieren lassen.

\section{Notes}\label{notes}

This demo file is intended~\cite{Tentschert2000} to serve as demo. I wish
you the best~\cite{li2014chaos} of success. test
test~\cite{gonzalez2017smells} \ldots{}

Building robots in hardware to test a swarm algorithm is possible, but
expensive
\href{https://en.wikipedia.org/wiki/Swarm_robotic_platforms}{hardware}

A hardware specific simulation of a single device exists, too:
\href{https://infocenter.nordicsemi.com/index.jsp?topic=\%2Fcom.nordic.infocenter.sdk52.v0.9.0\%2Fant_examples_ant_fs.html}{BLE
Ant client sim}

Using VM with router/device firmware and a network layout should be
interessting: \href{https://www.gns3.com/community}{Network sim}

\subsection{Ant algorithm}\label{ant-algorithm}

\begin{enumerate}
\def\labelenumi{\arabic{enumi}.}
\tightlist
\item
  clever
\end{enumerate}

\subsection{Bee algorithm}\label{bee-algorithm}

\begin{enumerate}
\def\labelenumi{\arabic{enumi}.}
\tightlist
\item
  Ein kleiner teil der Bienen (Scouts) sucht global nach ``gärten''
\item
  Da Gärten, wo es viel ``Futter'' gibt ein großes Areal sind, kommen
  viele Scouts zum dance-floor zurück mit ähnlichen Koordinaten
\item
  normale Bienen folgen nun den Scouts und verbreiten sich im Areal
  (Nahsuche)
\item
  ein Teil fliegt als Scout weiter und sucht neue Gärten
\item
  durch die Nahsuche werden schlechte Regionen von den Bienen
  ausgeschlossen und bessere Regionen ``gespeichert''
\item
  Biene, die bessere Region fand, wird neuer Scout und fliegt zum dance
  floor
\item
  Biene, die in der Region keine Verbesserung nach einer Weile fand,
  sucht neue Gärten
\end{enumerate}

\section{Related Work}\label{related-work}

intro about solutions and our special new solution/idea. iter through
related work with focus on different solutions. Why are some aspects
open?

\section{Methods}\label{methods}

Describe technique, structure and data collection of our solution.

\section{Results}\label{results}

intro about selected data, getting them and how we analyse them

\subsection{Definitions and Taxonomy}\label{definitions-and-taxonomy}

Parameters

\begin{itemize}
\tightlist
\item
  different food area size
\item
  different distance between food area and home
\item
  different count of food areas
\item
  different direction random size
\item
  different prefer pattern for dusty tiles
\item
  different amount, how strong the way decisions effects on setting the
  new dust
\item
  better to add or multiply dust on a tile
\item
  loosing dust on tiles rate
\item
  different hunger limits (?)
\item
  different random seeds (?)
\end{itemize}

Effects

\begin{itemize}
\tightlist
\item
  development of collected food (\%) in 1000 iterations
\item
  used dust (sum) over 1000 iterations
\item
  how many died (how long did they life)
\item
  development of random-, jam- and dust-decisions (\%) in a simulation
\item
  distance after last 10 steps before dying (ant traps?)
\end{itemize}

\subsection{Aspect 1}\label{aspect-1}

Subsection text here.

\subsection{Aspect 2}\label{aspect-2}

Subsection text here.

\subsection{Discussion}\label{discussion}

intro, offer explanation and reference to literature

\section{Conclusion}\label{conclusion}

The conclusion goes here.

\textbf{Future Work}: new open questions? how can we find answers in the
future? How can we use our solutions in the future?
