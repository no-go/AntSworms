
%% bare_conf.tex
%% V1.4b
%% 2015/08/26
%% by Michael Shell
%% See:
%% http://www.michaelshell.org/
%% for current contact information.
%%
%% This is a skeleton file demonstrating the use of IEEEtran.cls
%% (requires IEEEtran.cls version 1.8b or later) with an IEEE
%% conference paper.
%%
%% Support sites:
%% http://www.michaelshell.org/tex/ieeetran/
%% http://www.ctan.org/pkg/ieeetran
%% and
%% http://www.ieee.org/

%%*************************************************************************
%% Legal Notice:
%% This code is offered as-is without any warranty either expressed or
%% implied; without even the implied warranty of MERCHANTABILITY or
%% FITNESS FOR A PARTICULAR PURPOSE! 
%% User assumes all risk.
%% In no event shall the IEEE or any contributor to this code be liable for
%% any damages or losses, including, but not limited to, incidental,
%% consequential, or any other damages, resulting from the use or misuse
%% of any information contained here.
%%
%% All comments are the opinions of their respective authors and are not
%% necessarily endorsed by the IEEE.
%%
%% This work is distributed under the LaTeX Project Public License (LPPL)
%% ( http://www.latex-project.org/ ) version 1.3, and may be freely used,
%% distributed and modified. A copy of the LPPL, version 1.3, is included
%% in the base LaTeX documentation of all distributions of LaTeX released
%% 2003/12/01 or later.
%% Retain all contribution notices and credits.
%% ** Modified files should be clearly indicated as such, including  **
%% ** renaming them and changing author support contact information. **
%%*************************************************************************


% *** Authors should verify (and, if needed, correct) their LaTeX system  ***
% *** with the testflow diagnostic prior to trusting their LaTeX platform ***
% *** with production work. The IEEE's font choices and paper sizes can   ***
% *** trigger bugs that do not appear when using other class files.       ***
% The testflow support page is at:
% http://www.michaelshell.org/tex/testflow/



\documentclass[conference]{meta/IEEEtran}
% Some Computer Society conferences also require the compsoc mode option,
% but others use the standard conference format.
%
% If IEEEtran.cls has not been installed into the LaTeX system files,
% manually specify the path to it like:
% \documentclass[conference]{../sty/IEEEtran}

\usepackage[T1]{fontenc} % utf8
\usepackage[utf8]{inputenc} % utf8

% -----------------------------------------------------------------------------------
%\usepackage{german} % Zusammenfassung, Literatur anstelle von Abstract,References
%\usepackage[german]{babel} % deutsche silbentrennung
% -----------------------------------------------------------------------------------


% Some very useful LaTeX packages include:
% (uncomment the ones you want to load)



% *** MISC UTILITY PACKAGES ***
%
%\usepackage{ifpdf}
% Heiko Oberdiek's ifpdf.sty is very useful if you need conditional
% compilation based on whether the output is pdf or dvi.
% usage:
% \ifpdf
%   % pdf code
% \else
%   % dvi code
% \fi
% The latest version of ifpdf.sty can be obtained from:
% http://www.ctan.org/pkg/ifpdf
% Also, note that IEEEtran.cls V1.7 and later provides a builtin
% \ifCLASSINFOpdf conditional that works the same way.
% When switching from latex to pdflatex and vice-versa, the compiler may
% have to be run twice to clear warning/error messages.



% *** CITATION PACKAGES ***
%
\usepackage{cite}
% cite.sty was written by Donald Arseneau
% V1.6 and later of IEEEtran pre-defines the format of the cite.sty package
% \cite{} output to follow that of the IEEE. Loading the cite package will
% result in citation numbers being automatically sorted and properly
% "compressed/ranged". e.g., [1], [9], [2], [7], [5], [6] without using
% cite.sty will become [1], [2], [5]--[7], [9] using cite.sty. cite.sty's
% \cite will automatically add leading space, if needed. Use cite.sty's
% noadjust option (cite.sty V3.8 and later) if you want to turn this off
% such as if a citation ever needs to be enclosed in parenthesis.
% cite.sty is already installed on most LaTeX systems. Be sure and use
% version 5.0 (2009-03-20) and later if using hyperref.sty.
% The latest version can be obtained at:
% http://www.ctan.org/pkg/cite
% The documentation is contained in the cite.sty file itself.






% *** GRAPHICS RELATED PACKAGES ***
%
\ifCLASSINFOpdf
  % \usepackage[pdftex]{graphicx}
  % declare the path(s) where your graphic files are
  % \graphicspath{{../pdf/}{../jpeg/}}
  % and their extensions so you won't have to specify these with
  % every instance of \includegraphics
  % \DeclareGraphicsExtensions{.pdf,.jpeg,.png}
\else
  % or other class option (dvipsone, dvipdf, if not using dvips). graphicx
  % will default to the driver specified in the system graphics.cfg if no
  % driver is specified.
  % \usepackage[dvips]{graphicx}
  % declare the path(s) where your graphic files are
  % \graphicspath{{../eps/}}
  % and their extensions so you won't have to specify these with
  % every instance of \includegraphics
  % \DeclareGraphicsExtensions{.eps}
\fi
% graphicx was written by David Carlisle and Sebastian Rahtz. It is
% required if you want graphics, photos, etc. graphicx.sty is already
% installed on most LaTeX systems. The latest version and documentation
% can be obtained at: 
% http://www.ctan.org/pkg/graphicx
% Another good source of documentation is "Using Imported Graphics in
% LaTeX2e" by Keith Reckdahl which can be found at:
% http://www.ctan.org/pkg/epslatex
%
% latex, and pdflatex in dvi mode, support graphics in encapsulated
% postscript (.eps) format. pdflatex in pdf mode supports graphics
% in .pdf, .jpeg, .png and .mps (metapost) formats. Users should ensure
% that all non-photo figures use a vector format (.eps, .pdf, .mps) and
% not a bitmapped formats (.jpeg, .png). The IEEE frowns on bitmapped formats
% which can result in "jaggedy"/blurry rendering of lines and letters as
% well as large increases in file sizes.
%
% You can find documentation about the pdfTeX application at:
% http://www.tug.org/applications/pdftex





% *** MATH PACKAGES ***
%
%\usepackage{amsmath}
% A popular package from the American Mathematical Society that provides
% many useful and powerful commands for dealing with mathematics.
%
% Note that the amsmath package sets \interdisplaylinepenalty to 10000
% thus preventing page breaks from occurring within multiline equations. Use:
%\interdisplaylinepenalty=2500
% after loading amsmath to restore such page breaks as IEEEtran.cls normally
% does. amsmath.sty is already installed on most LaTeX systems. The latest
% version and documentation can be obtained at:
% http://www.ctan.org/pkg/amsmath





% *** SPECIALIZED LIST PACKAGES ***
%
%\usepackage{algorithmic}
% algorithmic.sty was written by Peter Williams and Rogerio Brito.
% This package provides an algorithmic environment fo describing algorithms.
% You can use the algorithmic environment in-text or within a figure
% environment to provide for a floating algorithm. Do NOT use the algorithm
% floating environment provided by algorithm.sty (by the same authors) or
% algorithm2e.sty (by Christophe Fiorio) as the IEEE does not use dedicated
% algorithm float types and packages that provide these will not provide
% correct IEEE style captions. The latest version and documentation of
% algorithmic.sty can be obtained at:
% http://www.ctan.org/pkg/algorithms
% Also of interest may be the (relatively newer and more customizable)
% algorithmicx.sty package by Szasz Janos:
% http://www.ctan.org/pkg/algorithmicx




% *** ALIGNMENT PACKAGES ***
%
%\usepackage{array}
% Frank Mittelbach's and David Carlisle's array.sty patches and improves
% the standard LaTeX2e array and tabular environments to provide better
% appearance and additional user controls. As the default LaTeX2e table
% generation code is lacking to the point of almost being broken with
% respect to the quality of the end results, all users are strongly
% advised to use an enhanced (at the very least that provided by array.sty)
% set of table tools. array.sty is already installed on most systems. The
% latest version and documentation can be obtained at:
% http://www.ctan.org/pkg/array


% IEEEtran contains the IEEEeqnarray family of commands that can be used to
% generate multiline equations as well as matrices, tables, etc., of high
% quality.




% *** SUBFIGURE PACKAGES ***
%\ifCLASSOPTIONcompsoc
%  \usepackage[caption=false,font=normalsize,labelfont=sf,textfont=sf]{subfig}
%\else
%  \usepackage[caption=false,font=footnotesize]{subfig}
%\fi
% subfig.sty, written by Steven Douglas Cochran, is the modern replacement
% for subfigure.sty, the latter of which is no longer maintained and is
% incompatible with some LaTeX packages including fixltx2e. However,
% subfig.sty requires and automatically loads Axel Sommerfeldt's caption.sty
% which will override IEEEtran.cls' handling of captions and this will result
% in non-IEEE style figure/table captions. To prevent this problem, be sure
% and invoke subfig.sty's "caption=false" package option (available since
% subfig.sty version 1.3, 2005/06/28) as this is will preserve IEEEtran.cls
% handling of captions.
% Note that the Computer Society format requires a larger sans serif font
% than the serif footnote size font used in traditional IEEE formatting
% and thus the need to invoke different subfig.sty package options depending
% on whether compsoc mode has been enabled.
%
% The latest version and documentation of subfig.sty can be obtained at:
% http://www.ctan.org/pkg/subfig




% *** FLOAT PACKAGES ***
%
%\usepackage{fixltx2e}
% fixltx2e, the successor to the earlier fix2col.sty, was written by
% Frank Mittelbach and David Carlisle. This package corrects a few problems
% in the LaTeX2e kernel, the most notable of which is that in current
% LaTeX2e releases, the ordering of single and double column floats is not
% guaranteed to be preserved. Thus, an unpatched LaTeX2e can allow a
% single column figure to be placed prior to an earlier double column
% figure.
% Be aware that LaTeX2e kernels dated 2015 and later have fixltx2e.sty's
% corrections already built into the system in which case a warning will
% be issued if an attempt is made to load fixltx2e.sty as it is no longer
% needed.
% The latest version and documentation can be found at:
% http://www.ctan.org/pkg/fixltx2e


%\usepackage{stfloats}
% stfloats.sty was written by Sigitas Tolusis. This package gives LaTeX2e
% the ability to do double column floats at the bottom of the page as well
% as the top. (e.g., "\begin{figure*}[!b]" is not normally possible in
% LaTeX2e). It also provides a command:
%\fnbelowfloat
% to enable the placement of footnotes below bottom floats (the standard
% LaTeX2e kernel puts them above bottom floats). This is an invasive package
% which rewrites many portions of the LaTeX2e float routines. It may not work
% with other packages that modify the LaTeX2e float routines. The latest
% version and documentation can be obtained at:
% http://www.ctan.org/pkg/stfloats
% Do not use the stfloats baselinefloat ability as the IEEE does not allow
% \baselineskip to stretch. Authors submitting work to the IEEE should note
% that the IEEE rarely uses double column equations and that authors should try
% to avoid such use. Do not be tempted to use the cuted.sty or midfloat.sty
% packages (also by Sigitas Tolusis) as the IEEE does not format its papers in
% such ways.
% Do not attempt to use stfloats with fixltx2e as they are incompatible.
% Instead, use Morten Hogholm'a dblfloatfix which combines the features
% of both fixltx2e and stfloats:
%
% \usepackage{dblfloatfix}
% The latest version can be found at:
% http://www.ctan.org/pkg/dblfloatfix



% *** PDF, URL AND HYPERLINK PACKAGES ***
%
\usepackage{url}
% url.sty was written by Donald Arseneau. It provides better support for
% handling and breaking URLs. url.sty is already installed on most LaTeX
% systems. The latest version and documentation can be obtained at:
% http://www.ctan.org/pkg/url
% Basically, \url{my_url_here}.

% *** Do not adjust lengths that control margins, column widths, etc. ***
% *** Do not use packages that alter fonts (such as pslatex).         ***
% There should be no need to do such things with IEEEtran.cls V1.6 and later.
% (Unless specifically asked to do so by the journal or conference you plan
% to submit to, of course. )

% correct bad hyphenation here
\hyphenation{op-tical net-works semi-conduc-tor}

\begin{document}

% paper title
% Titles are generally capitalized except for words such as a, an, and, as,
% at, but, by, for, in, nor, of, on, or, the, to and up, which are usually
% not capitalized unless they are the first or last word of the title.
% Linebreaks \\ can be used within to get better formatting as desired.
% Do not put math or special symbols in the title.


\title{Simulate Bug Algorithm}


% author names and affiliations
% use a multiple column layout for up to three different
% affiliations

\author{\IEEEauthorblockN{Jochen Peters}

\IEEEauthorblockA{Heinrich-Heine-Universität Düsseldorf\\
Institut für Informatik\\
40225 Düsseldorf, Germany\\
Email: jochen.peters@hhu.de}
\and
\IEEEauthorblockN{Homer Simpson}
\IEEEauthorblockA{Twentieth Century Fox\\
Springfield, USA\\
Email: homer@thesimpsons.com}}


% conference papers do not typically use \thanks and this command
% is locked out in conference mode. If really needed, such as for
% the acknowledgment of grants, issue a \IEEEoverridecommandlockouts
% after \documentclass

% for over three affiliations, or if they all won't fit within the width
% of the page, use this alternative format:
% 
%\author{\IEEEauthorblockN{Michael Shell\IEEEauthorrefmark{1},
%Homer Simpson\IEEEauthorrefmark{2},
%James Kirk\IEEEauthorrefmark{3}, 
%Montgomery Scott\IEEEauthorrefmark{3} and
%Eldon Tyrell\IEEEauthorrefmark{4}}
%\IEEEauthorblockA{\IEEEauthorrefmark{1}School of Electrical and Computer Engineering\\
%Georgia Institute of Technology,
%Atlanta, Georgia 30332--0250\\ Email: see http://www.michaelshell.org/contact.html}
%\IEEEauthorblockA{\IEEEauthorrefmark{2}Twentieth Century Fox, Springfield, USA\\
%Email: homer@thesimpsons.com}
%\IEEEauthorblockA{\IEEEauthorrefmark{3}Starfleet Academy, San Francisco, California 96678-2391\\
%Telephone: (800) 555--1212, Fax: (888) 555--1212}
%\IEEEauthorblockA{\IEEEauthorrefmark{4}Tyrell Inc., 123 Replicant Street, Los Angeles, California 90210--4321}}

\maketitle
% As a general rule, do not put math, special symbols or citations
% in the abstract
\begin{abstract}
The technological development is often motivated by nature concepts.
For example molecular machines with a minimal logic acting like complex biological
proteins. We are thinking ahead: these molecular machines will work as a robot
swarm in the future. Therefore we need a simple logic for each machine.
We concentrated on the ant algorithm to find and collect food. This algorithm use
pheromones and the environment to control and communicate to each machine (ant).
We found out, that simulating a simple implementation of this algorithm leads
to ant mills and high frequented routes to empty food areas. In this paper 
we present a small modification of the ant
algorithm to protect a swarm from collective misconduct. We show, how
successful this modification is and compare this concept to
nature and other ant algorithm simulations.
\end{abstract}

% use the taxonomy data of IEEE for the keywords
\begin{IEEEkeywords}
multi-robot systems, simulation, pheromones, insects
\end{IEEEkeywords}


% For peer review papers, you can put extra information on the cover
% page as needed:
% \ifCLASSOPTIONpeerreview
% \begin{center} \bfseries EDICS Category: 3-BBND \end{center}
% \fi
%
% For peerreview papers, this IEEEtran command inserts a page break and
% creates the second title. It will be ignored for other modes.
\IEEEpeerreviewmaketitle

% An example of a floating figure using the graphicx package.
% Note that \label must occur AFTER (or within) \caption.
% For figures, \caption should occur after the \includegraphics.
% Note that IEEEtran v1.7 and later has special internal code that
% is designed to preserve the operation of \label within \caption
% even when the captionsoff option is in effect. However, because
% of issues like this, it may be the safest practice to put all your
% \label just after \caption rather than within \caption{}.
%
% Reminder: the "draftcls" or "draftclsnofoot", not "draft", class
% option should be used if it is desired that the figures are to be
% displayed while in draft mode.
%
%\begin{figure}[!t]
%\centering
%\includegraphics[width=2.5in]{myfigure}
% where an .eps filename suffix will be assumed under latex, 
% and a .pdf suffix will be assumed for pdflatex; or what has been declared
% via \DeclareGraphicsExtensions.
%\caption{Simulation results for the network.}
%\label{fig_sim}
%\end{figure}

% Note that the IEEE typically puts floats only at the top, even when this
% results in a large percentage of a column being occupied by floats.


% An example of a double column floating figure using two subfigures.
% (The subfig.sty package must be loaded for this to work.)
% The subfigure \label commands are set within each subfloat command,
% and the \label for the overall figure must come after \caption.
% \hfil is used as a separator to get equal spacing.
% Watch out that the combined width of all the subfigures on a 
% line do not exceed the text width or a line break will occur.
%
%\begin{figure*}[!t]
%\centering
%\subfloat[Case I]{\includegraphics[width=2.5in]{box}%
%\label{fig_first_case}}
%\hfil
%\subfloat[Case II]{\includegraphics[width=2.5in]{box}%
%\label{fig_second_case}}
%\caption{Simulation results for the network.}
%\label{fig_sim}
%\end{figure*}
%
% Note that often IEEE papers with subfigures do not employ subfigure
% captions (using the optional argument to \subfloat[]), but instead will
% reference/describe all of them (a), (b), etc., within the main caption.
% Be aware that for subfig.sty to generate the (a), (b), etc., subfigure
% labels, the optional argument to \subfloat must be present. If a
% subcaption is not desired, just leave its contents blank,
% e.g., \subfloat[].


% An example of a floating table. Note that, for IEEE style tables, the
% \caption command should come BEFORE the table and, given that table
% captions serve much like titles, are usually capitalized except for words
% such as a, an, and, as, at, but, by, for, in, nor, of, on, or, the, to
% and up, which are usually not capitalized unless they are the first or
% last word of the caption. Table text will default to \footnotesize as
% the IEEE normally uses this smaller font for tables.
% The \label must come after \caption as always.
%
%\begin{table}[!t]
%% increase table row spacing, adjust to taste
%\renewcommand{\arraystretch}{1.3}
% if using array.sty, it might be a good idea to tweak the value of
% \extrarowheight as needed to properly center the text within the cells
%\caption{An Example of a Table}
%\label{table_example}
%\centering
%% Some packages, such as MDW tools, offer better commands for making tables
%% than the plain LaTeX2e tabular which is used here.
%\begin{tabular}{|c||c|}
%\hline
%One & Two\\
%\hline
%Three & Four\\
%\hline
%\end{tabular}
%\end{table}


% Note that the IEEE does not put floats in the very first column
% - or typically anywhere on the first page for that matter. Also,
% in-text middle ("here") positioning is typically not used, but it
% is allowed and encouraged for Computer Society conferences (but
% not Computer Society journals). Most IEEE journals/conferences use
% top floats exclusively. 
% Note that, LaTeX2e, unlike IEEE journals/conferences, places
% footnotes above bottom floats. This can be corrected via the
% \fnbelowfloat command of the stfloats package.

% --------------------------------------------------------------------------


\begin{IEEEkeywords}
simulation, robots, ants, bugs
\end{IEEEkeywords}

\section{Introduction (not ready)}\label{introduction-not-ready}

Raketensteuerungen von Atombomben werden oft von pneumatischen
(Staubsaugermotoren und Klappen wie bei Zuse I) oder hydraulischen
Systemen betrieben, damit sie bei einem elektromagnitischen Impuls einer
anderen Bombe noch weiter funktionieren. In der Atmospähre eines anderen
Planeten könnte eine Elektronik z.B. wegen der starken statischen Ladung
des Marsstaubs ähnliche Probleme verursachen. Unter Wasser oder in einem
sandigem Medium können GPS und Funkwellen kaum bis gar nicht genutzt
werden. Schaut man sich aktuelle Nanotechnologien an, so werden
chemische Botenstoffe zur Auslösung bestimmter Reaktionen,
interessanter, als mit Funktechnik Roboter in unserer Blutbahn
kommunizieren zu lassen. In der Natur werden bereits Hormone (und keine
Funkwellen) als Kommunikation eingesetzt. Sollte man chemische Propeller
an Makromolekülen bilden oder betreiben, die durch eine chemische
Reaktion aufgrund einer bestimmten Konzentration eines künstlichen
Hormons an der Stelle des Moleküls, an dem der Propeller sitzt, so sind
logische Operationen oder gar Speichermedien in einem Molekül begrenzt.
Durch künstliche Methabolismen eine turingvollständige Sprache bilden,
die als Eingabe Konzentrationen von bestimmten Molekülen akzeptiert und
gleiches als Ausgabemedium zur Steuerung einer molekularen Struktur
erzeugt, mag heute noch nach Science Fiction klingen - findet aber
bereits in Laboren statt. (todo: hier fehlt mir noch eine Referenz)

In contrast to bees, the simulated ants did not split into a searching
and a collecting behavior in that case.

We added a simple traffic jam control to each unit and measured a better
food supply.

This demo file is intended~\cite{Tentschert2000} to serve as demo. I wish
you the best~\cite{li2014chaos} of success. test
test~\cite{gonzalez2017smells} \ldots{}

Molecular machines, optical digital processing units and analogue
computers with a pneumatic logical unit are able to work in special
environments, where modern micro controllers are too large, use too much
electric energy or are not resistant to massive electromagnetic forces.
Many of these techniques are unable to use wireless communication,
unable to store big data sizes or do millions of instructions of a
complex algorithm. If you want to solve a big problem with it, you have
to solve it in a collective opportunistic way with a simple logic for
each unit.

Primitive Algorithmen wie den Ant-Algorithmus sind bereits seit vielen
Jahren bekannt. Dieser nutzt einen Duftstoff, der in einer Region von
jedem Teilnehmer freigesetzt wird, um eine Art kollektiven Speicher zu
verwirklichen. Ziel ist es auf diese Art Spuren für andere Teilnehmer zu
hinterlassen, welche zu Futterquellen führen. Dieses Futter soll dann
ins Nest getragen werden. Je mehr Ameisen auf einer gelegten Spur
laufen, desto stärker wird diese Spur. Theoretische Modelle zeigen, dass
sich dadurch sogar kürzere Wege zurück ins Nest durchsetzen, da diese
Wege von den Ameisen, die ihn gefunden haben, häufiger frequentiert wird
- die Duftspur wird so verstärkt. In der Literatur finden sich viele
Variationen und theoretische Modelle, wie ein solcher Algorithmus, durch
die Natur motiviert, nachgebildet werden kann. Z.B. sind lokale
Algorithmen (verteilte Algorithmen mit linearer Laufzeit, die nicht von
der Größe des Netzes abhängig sind), einem Schwarm-Algorithmus wie dem
Ant-Algorithmus sehr ähnlich, betrachten aber keine kontinuierliche
Veränderung der Daten Aufgrund von Bewegungen. Schaut man sich in der
Literatur die Ant-Algorithmen im Detail an, so wird der Erfolg einer
Ameisen-Kolonie bei gewählten Parametern, die das Legen der Duftspuren
und deren ``Verwitterung'' beeinflussen, kaum behandelt.

\section{Related Work (not ready)}\label{related-work-not-ready}

intro about solutions and our special new solution/idea. iter through
related work with focus on different solutions. Why are some aspects
open?

\section{Methods}\label{methods}

The simulation of our ant algorithm was splitted into the following
parts:

\begin{itemize}
\tightlist
\item
  a simulator (doing 1000 loops)
\item
  a world (2 dimensional)
\item
  food tiles (marked as food with a nutritive value)
\item
  pheromone tiles
\item
  30 robot objects (each with our ant algorithm)
\end{itemize}

In this chapter we explain each part and who we initialized it with
fixed values. The last chapter is about our ant algorithm and the
traffic jam control to prevent ant mills.

\subsection{Simulated parts and their
interaction}\label{simulated-parts-and-their-interaction}

We did not want to print any code here to make it easy to implement the
parts, concept and objects in your preferred system, language or medium.

\subsubsection{World}\label{world}

Our world is a 2 dimensional hex grid. Thus our robots and tiles can
only placed into that grid. The worlds are generated with 10 different
random seeds to place food tiles with different area sizes, positions
and nutritive values. When we measured a new result with a different
parameter, we simulated it with the same 10 different worlds and build a
cumulative value.

This hex grid is placed on a normal \((x,y)\) coordinate system, but the
placement of each robot and tile must fit into the hex grid. If we talk
about distances we use the euclidean distance based on that x and y
values.

\subsubsection{Robots (ant)}\label{robots-ant}

We initialized the simulator with 30 ants placed on \((0,0)\) in the
world (we call it \emph{home}). Thus they start with a traffic jam. Each
ant has a \emph{hunger level}. If an ant does not reach any food or
carries it to home, the hunger level is increment by one. If the limit
reaches \textbf{50}, it dies immediately and respawns at home. Every ant
is awaked by each simulator iteration and runs our ant algorithm with
traffic jam control.

Because of the hex grid, the ant can only move to 6 neighbor fields. For
Example if the ant starts at \((0,0)\) and is awaked by the simulator,
it can make a step to \((-1.0, 0.0)\), \((1.0, 0.0)\), \((-0.5, 1.0)\),
\((-0.5, -1.0)\), \((0.5, 1.0)\) or \((0.5, -1.0)\).

\subsubsection{Food tiles}\label{food-tiles}

Every food tile is part of an food area. In a world generated by a seed
these areas can have a size between 1 and 16 tiles. A world can have up
to \(30\) areas. Each tile is generated with a nutritive value between
\(1\) and \(20\). If an ant reaches the food tile, the nutritive value
is decremented by one. If the nutritive value is zero, the food tile is
deleted immediately. The initial placement of the areas is handled by a
random seed. There are two different concepts placing them with a
maximal area distance of \(20\) (\(\pm\) 5):

\begin{itemize}
\tightlist
\item
  spray
\item
  ordered
\end{itemize}

The spray concept setting an area on a random position and place the
next area randomly (but not to far away). Finally the sprayed areas are
centered in the middle of the world.

The ordered concept is a grid arranged around the ant home location. The
position is very by \(5\).

We decided to choose these concept to get random worlds with reachable
food for the ants. Vary the maximal distance of the ant and/or vary the
maximal distance of each area may focused in future work.

\subsubsection{Pheromone tiles}\label{pheromone-tiles}

The pheromone tiles are generated by ants, if an ant finds food and
running back to home. This is the only way creating tiles. We store the
pheromone potency internally in an alpha value with \(1.0\) as maximum.
The ants are able to increment this value. A tile with a \emph{potency}
less than \(0.0001\) will be removed by the simulator. You can define it
as a detection level of an ant, if you want.

After all ants do their work after a simulator iteration, the tiles will
be deleted or the potency will be reduced by a \emph{evaporation}:

\[ potency_{i+1} = \frac{100 - evaporation}{100} * potency_{i} \]

\subsubsection{Simulator}\label{simulator}

The Simulator loads the world with it special seed and iterated \(1000\)
times through \(30\) ants. Thus a single ant is being \(wakeup()\) for
\(1000\) times to walk, detect, collect or setting/changing a pheromone
tile.

\subsection{Ant algorithm with traffic jam
control}\label{ant-algorithm-with-traffic-jam-control}

todo \ldots{}

\section{Results (not ready)}\label{results-not-ready}

intro about selected data, getting them and how we analyse them

\subsection{Definitions and Taxonomy}\label{definitions-and-taxonomy}

Parameters

\begin{itemize}
\tightlist
\item
  different food area size
\item
  different distance between food area and home
\item
  different count of food areas
\item
  different direction random size
\item
  different prefer pattern for dusty tiles
\item
  different amount, how strong the way decisions effects on setting the
  new dust
\item
  better to add or multiply dust on a tile
\item
  loosing dust on tiles rate
\item
  different hunger limits (?)
\item
  different random seeds (?)
\end{itemize}

Effects

\begin{itemize}
\tightlist
\item
  development of collected food (\%) in 1000 iterations
\item
  used dust (sum) over 1000 iterations
\item
  how many died (how long did they life)
\item
  development of random-, jam- and dust-decisions (\%) in a simulation
\item
  distance after last 10 steps before dying (ant traps?)
\end{itemize}

\subsection{Aspect 1}\label{aspect-1}

Subsection text here.

\subsection{Aspect 2}\label{aspect-2}

Subsection text here.

\subsection{Discussion}\label{discussion}

intro, offer explanation and reference to literature

\section{Conclusion (not ready)}\label{conclusion-not-ready}

Ebenso gibt es beim Ant-Algorithmus einen Aspekt, der bei der ältesten
Art, den Wanderameisen, besondere Aufmerksamkeit verdient:
``Ameisenmülen''. In diesem Fall laufen alle Ameisen im Kreis und
sterben an Erschöpfung, da sie so keine weiteren Futerquellen finden.
Ein weiterer Punkt, der einen Erfolg eines zu primitiven Ant-Algorithmus
in Frage stellt, sind Ameisen-Straßen, die zu versiegten Futterquellen
führen. Wir konnten in unseren Simulationen beide ``Fehlentwicklungen''
beobachten.

Man kann den Ant-Algorithmus durch Hinzunahme von GPS, Funktechnik und
einem großen Arbeitsspeicher zur modifizieren, um ein solches
Fehlverhalten zu erkennen oder zu verhindern, aber diese Modifikationen
setzen folgendes voraus:

\begin{itemize}
\tightlist
\item
  elektrische Energie
\item
  Rechenleistung
\item
  mindestens die Größe des Chips
\item
  bedenkenlose Anwendung von Funkwellen
\item
  Speichermedium
\end{itemize}

Es ist also nicht vollkommen abwegig, einen Ant- Algorithmus so simple
wie möglich zu halten, der die Umge- bung als Massenspeicher nutzt,
welche mit dem Operations- gebiet mitwächst. Wir haben diesen Aspekt der
Einfachheit in Hinblick auf Erfolg und Robustheit bzgl. äußerer
Parameter untersucht und kommen zu dem Ergebnis, dass auch andere
natürliche Aspekte der Wanderameise eine wichtige Rolle Spielen müssen,
was diese Spezies so lange parallel zu anderen Ameisenarten hat
erfolgreich koexistieren lassen.

\textbf{Future Work}: new open questions? how can we find answers in the
future? How can we use our solutions in the future?


%\section*{Acknowledgment}
The authors would like to thank...


% trigger a \newpage just before the given reference
% number - used to balance the columns on the last page
% adjust value as needed - may need to be readjusted if
% the document is modified later
%\IEEEtriggeratref{8}
% The "triggered" command can be changed if desired:
%\IEEEtriggercmd{\enlargethispage{-5in}}

% references section

% can use a bibliography generated by BibTeX as a .bbl file
% BibTeX documentation can be easily obtained at:
% http://mirror.ctan.org/biblio/bibtex/contrib/doc/
% The IEEEtran BibTeX style support page is at:
% http://www.michaelshell.org/tex/ieeetran/bibtex/
%\bibliographystyle{IEEEtran}
% argument is your BibTeX string definitions and bibliography database(s)
%\bibliography{IEEEabrv,../bib/paper}
%
% <OR> manually copy in the resultant .bbl file
% set second argument of \begin to the number of references
% (used to reserve space for the reference number labels box)
\begin{thebibliography}{1}

\bibitem{grappa:nelson}
J.~Nelson, B.~Holt, B.~Myers, P.~Briggs, L.~Ceze, S.~Kahan and M.~Oskin.
 (2014, April) 
 \emph{Grappa: A Latency-Tolerant Runtime for Large-Scale Irregular Applications}.
 [Online].
 Available: http://grappa.io/
\end{thebibliography}



\end{document}
